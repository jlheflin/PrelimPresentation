\section{NWChemEx}

\begin{frame}
\sectionpage
\end{frame}

\begin{frame}
\frametitle{Not Another QM Packge! Why!?}
Has you or a loved one tried to develop quantum chemistry models only to find
that you need to know how to write Fortran, compile it against the original
project, and ensure that it works in the end, only to have the entire project
not work after removing your new code and left at square 0 with no quantum
chemistry package to use? That's basically a story I have heard from every
physical chemist that I have talked to. It is already hard to find and use
the tools for computational chemistry. To develop new tools with zero computer
science and programming background? Nearly impossible (Ask me how I know).
\end{frame}

\begin{frame}
Therein lies the problem: Every time a chemist decides to develop a new theory
the barrier for entry is huge. The NWChemEx Project is a solution to that
issue, or is at least trying to be. The main workhorse of the project is the
SimDE project, or the Simulation Development Environment project. This part
defines the basic data structures for the NWChemEx project. 
\end{frame}

\begin{frame}

\end{frame}

\begin{frame}
\frametitle{Slide 2 Title}
Content for slide 2
\end{frame}
