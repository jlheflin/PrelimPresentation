\section{Designing Preorganized Chelators for Nuclear Fission Products}

\begin{frame}
\frametitle{Motivation}
\begin{itemize}
	\item 
\end{itemize}
\section{Designing Preorganized Chelators for Nuclear Fission Product Separations}

\begin{frame}
\sectionpage
\end{frame}

\begin{frame}
\frametitle{Why do we care?}
Being able to separate out nuclear fission products allows for characterization of these species that allows for forensic information to be obtained. This can be done in several ways, but the only way that I can think of right now is
that the concentrations of the nuclear fission products are statistically related to the source nuclear material that is used. Being able to identify the source material allows for better bracktracking to the source. Other reasons for
separating out radioactive products include removal from patients within the medical field. If these radioactive species are not removed from the patient, they can be deposited into bones as a calcium replacement, causing further
pysiological compilications. 
\end{frame}

\begin{frame}
Current separation procedures for nuclear fission products include resin based separations, and others that are being used within the medical field. 
\end{frame}


\begin{frame}
\frametitle{Goals}
Content for slide 2
\end{frame}


\begin{frame}
\frametitle{Current Work}
\end{frame}
